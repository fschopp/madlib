\documentclass[letterpaper,11pt]{scrreprt}
\usepackage[american]{babel}
\usepackage[latin1]{inputenc}
\usepackage[T1]{fontenc}
\usepackage[top=1in,bottom=1in,left=1in,right=1in]{geometry}

\usepackage{amsmath}                  % Bsp. f�r Fraktur-Buchstaben, \mathfrak
\usepackage{amssymb}

\usepackage[hyperref]{ntheorem}
\usepackage[
	bookmarks,
	colorlinks=false,
	linkcolor=blue,
	citecolor=blue,
	pagebackref=false,
	pdftitle={MADlib Design Document},
	pdfauthor={Florian Schoppmann},
	pdfsubject={},
	pdfkeywords={}
]{hyperref}                            % F�r PDF-Features
\usepackage{csquotes}                  % Strongly recommended for biblatex
\usepackage[
	maxnames=2,
	firstinits=true
]{biblatex}
\usepackage{scrpage2}                  % Kopf und Fu�zeilen
\usepackage{color}                     % F�r Farben, eigtl. nur f�r \todo
\usepackage{enumitem}                  % Tolle enumerate-Umgebungen
\usepackage{ctable}
\usepackage{tabularx}
%\usepackage{rcs}                       % Support for RCS/CVS Keyword substitution
                                       % Commenting out will remove version control information
\usepackage{xspace}                    % Korrekte Leerzeichen nach \newcommand-Definitionen
\usepackage[noend]{algpseudocode}      % Algorithmen-Umgebungen
\usepackage{minted}


% BEGIN Doc Layout
	\allowdisplaybreaks[3]

	\pagestyle{scrheadings}
	\setkomafont{disposition}{\normalcolor\bfseries}
	\setkomafont{descriptionlabel}{\bfseries}
	\setkomafont{captionlabel}{\usekomafont{disposition}}
	
	\setlength{\arrayrulewidth}{.5pt}
	\numberwithin{equation}{section}
	\renewcommand{\theenumi}{\roman{enumi}}
	\renewcommand{\labelenumi}{\theenumi)}
	
	\newcommand{\otoprule}{\midrule[\heavyrulewidth]}
	
	\setcounter{secnumdepth}{3}
	
	\makeatletter
	% Algorithms are expected to have an optional argument of form
	% FunctionName$(ArgumentList)$, e.g., DiscreteSample$(A, w)$
	\def\internal@funcName#1$(#2)${#1}
	\newcommand\funcName[1]{\internal@funcName #1}
	\newtheoremstyle{algorithm}
		{\item[\rlap{\vbox{\hbox{\hskip\labelsep \theorem@headerfont
			##1\ ##2\theorem@separator}\hbox{\strut}}}]}%
		{\item[\rlap{\vbox{\hbox{\hskip\labelsep {\theorem@headerfont
			##1}\ \normalfont\texttt{##3}{\theorem@headerfont\theorem@separator}}\hbox{\strut}}}]%
			\def\@currentlabel{\texttt{\funcName{##3}}}}
	\makeatother
	
	\makeatletter
	% Also display JSTOR in small caps
	% http://sourceforge.net/tracker/index.php?func=detail&aid=3152938&group_id=244752&atid=1126006
	\DeclareFieldFormat{eprint:arxiv}{%
	  \textsc{arXiv}\addcolon
	  \ifhyperref
	    {\href{http://arxiv.org/\abx@arxivpath/#1}{%
	       \nolinkurl{#1}%
	       \iffieldundef{eprintclass}
		 {}
		 {\addspace\texttt{\mkbibbrackets{\thefield{eprintclass}}}}}}
	    {\nolinkurl{#1}
	     \iffieldundef{eprintclass}
	       {}
	       {\addspace\texttt{\mkbibbrackets{\thefield{eprintclass}}}}}}
	\DeclareFieldFormat{eprint:jstor}{%
	  \mkbibacro{JSTOR}\addcolon\space
	  \ifhyperref
	    {\href{http://www.jstor.org/stable/#1}{\nolinkurl{#1}}}
	    {\nolinkurl{#1}}}
	% Some conferences do not have DOIs for their papers, but they do get
	% IDs in the ACM Digital Library. E.g., SODA papers.
	\DeclareFieldFormat{eprint:acm}{%
	  \mkbibacro{ACM}\addcolon\space
	  \ifhyperref
	    {\href{http://dl.acm.org/citation.cfm?id=#1}{\nolinkurl{#1}}}
	    {\nolinkurl{#1}}}
	\makeatother
% END Doc Layout

% BEGIN General Definitions
	\newcommand{\todo}[1]{\textbf{\color{red}#1}}

	% BEGIN Mathematical Definition
		% Space (only) in displaymath (e.g., between mathematical expression and punctuation mark)
		\newcommand{\SiM}{\mathchoice{\,}{}{}{}}
	% END Mathematical Operators
	
	% BEGIN URLs
		\newcommand{\mailto}[1]{\href{mailto:#1}{\nolinkurl{#1}}}
		\newcommand{\doi}[1]{DOI: \href{http://dx.doi.org/#1}{\nolinkurl{#1}}}
	% END URLs
	
	\makeatletter
	% BEGIN Mathematical Definitions
		% BEGIN Set Symbols
			\newcommand{\setsymbol}[1]{\mathbb{#1}}
			\newcommand{\N}{\@ifstar{\setsymbol{N}_0}{\setsymbol{N}}}
			\newcommand{\R}{\setsymbol{R}}
		    \newcommand{\Nupto}{\@ifstar{\Nupto@star}{\Nupto@nostar}}
		    \newcommand{\Nupto@star}[1]{[#1]_0}
		    \newcommand{\Nupto@nostar}[1]{[#1]}
		% END Set Symbols
		\renewcommand{\vec}[1]{\ensuremath{\boldsymbol{#1}}}
	% END Mathematical Definitions
	\makeatother
	
	\renewcommand{\vec}[1]{\ensuremath{\boldsymbol{#1}}}
	\newcommand{\enumref}[1]{(\ref{#1})}
	
	\makeatletter
	\newcommand{\symlabel}[2]{\def\@currentlabel{\texttt{#1}}\texttt{#1}\label{#2}}
	\makeatother
	
	% BEGIN Algorithms
	\theoremstyle{algorithm}
	\theorembodyfont{\upshape}
	\newtheorem{algorithm}{Algorithm}[section]
	
	\newlength{\alglabelwidth}
	\newcommand{\alginput}[1]{%
		\par\noindent%
		\settowidth{\alglabelwidth}{\emph{Output:}}%
		\makebox[\alglabelwidth][l]{\emph{Input:}} \begin{tabular}[t]{l} #1 \end{tabular}}
	\newcommand{\algoutput}[1]{%
		\par\noindent%
		\settowidth{\alglabelwidth}{\emph{Output:}}%
		\makebox[\alglabelwidth][l]{\emph{Output:}} \begin{tabular}[t]{l} #1 \end{tabular}}
	\newcommand{\algprecond}[1]{%
		\par\noindent\textit{Initialization/Precondition: #1}}
		
	\newcommand{\set}{\leftarrow}
	\DeclareMathOperator{\random}{random}
	\newcommand{\dist}{\ensuremath{\mathit{dist}}}
	\newcommand{\List}{\mathrm{List}}
	\newcommand{\Sample}{\mathit{Sample}}
	\algblockdefx[With]{With}{EndWith}%
		[1]{\textbf{with} #1 \textbf{do}}%
		[0]{End}
	\algnotext[With]{EndWith}
	% END Algorithms
	
	% BEGIN Listing environments
	\newminted{sql}{mathescape,
					numbersep=5pt,
					tabsize=4,
					gobble=8,
					framesep=2mm,
					xleftmargin=\leftmargini
					}
	% END Listing environments
% END General Definitions

\bibliography{../literature.bib}

% BEGIN Preamble
\title{%
	MADlib Design Document%
}

\begin{document}

\maketitle

\tableofcontents

\chapter{Sampling}

\section{Sampling without Replacement} \label{sec:SampingWOReplacement}

Given a list of known size $n$ through that we can iterate with arbitrary increments, sampling $m$ elements without replacement can be implemented in time $O(m)$, i.e., proportional to the sample size and independent of the list size \cite{V84a}. Even if the list size $n$ is unknown in advance, sampling can still be implemented in time $O(m(1 + \log \frac nm))$ \cite{V85a}.

While being able to iterate through a list with arbitrary increments might seem like a very modest requirement, it is still not always an option in real-world databases (e.g., in PostgreSQL). It is therefore important to also consider more constrained algorithms.

\subsection{Probabilistic Sampling}

Probabilistic sampling selects each element in the list with probability $p$. Hence, the sample size is a random variable with Binomial distribution $B(n, p)$ and expectation $np$. The standard deviation is $\sqrt{np(1 - p)}$, i.e., approximately $\sqrt{np}$ if $p$ is small. In many applications a fixed sample size $m$ is needed, however. In this case, we could choose $p$ slightly larger than $m/n$, so that with high probability at least $m$ items are selected. Items in excess of $m$ are then discarded.

\subsubsection{Formal Description}

In the following, we discuss how to choose $p$ so that with high probability at least $m$ elements are sampled, but also not ``much'' more than $m$ (in fact, only $O(\sqrt m)$ more in expectation).

In mathematical terms: What is a lower bound on the probability $p$ so that for a random variable $X \sim B(n,p)$ we have that $\Pr[X < m] \leq \epsilon$? We use the Chernoff bound for a fairly good estimate. It says
%
\begin{align*}
	\Pr[X < (1 - \delta) \cdot \mu] \leq \exp\left( \frac{-\delta^2}{2} \cdot \mu \right)
	\SiM,
\end{align*}
where $\mu = np$ is the expectation of $X$, and $\delta \geq 0$. We set $m = (1 - \delta) \cdot \mu$, or equivalently $\delta = \frac{\mu - m}{\mu}$.
%
This yields
\begin{align}
	\Pr[X < m] \leq \exp\left( \frac{-(\mu - m)^2}{2 \mu} \right)
	\SiM. \label{eq:SamplingWOReplacent:GP:2}
\end{align}
%
We want the right-hand side of \eqref{eq:SamplingWOReplacent:GP:2} to be bounded by $\epsilon$ from above. Rearranging this gives
\begin{align*}
	\mu \geq m - \ln(\epsilon) + \sqrt{\ln^2(\epsilon) - 2 m \ln(\epsilon)}
	\SiM.
\end{align*}
Since $p = \mu / n$, this immediately translates into a lower bound for $p$. For instance, suppose we require $\epsilon = 10^{-6}$, i.e., we want the probability of our sample being too small to be less than one in a million. $\ln(10^{-6}) \approx -13.8$, so we could choose
\begin{align*}
	p \geq \frac{m + 14 + \sqrt{196 + 28m}}{n}
	\SiM.
\end{align*}

Note that the bound on $\mu$ does not depend on $n$. So in expectation, only $O(m + \sqrt m)$ items are selected. At the same time, at least $m$ items are selected with very high probability.

\subsubsection{Implementation in SQL}

In real-world DBMSs, probabilistic sampling has the advantage that it is trivially data-parallel. Discarding excessive items can be done using the well-known \texttt{ORDER BY random() LIMIT} idiom. Tests show that PostgreSQL is very efficient in doing the sorting (today's CPUs can easily sort 1 million numbers in less than a couple hundred milliseconds). In fact, the sorting cost is almost not measurable if the sample size is only at the scale of several million or less. Since \texttt{ORDER BY random() LIMIT} is an often-used idiom, there is also hope that advanced optimizers might give it special treatment. Put together, in order to sample $m$ random rows uniformly at random, we write:
\begin{sqlcode}
	SELECT * FROM list WHERE random() < p ORDER BY random() LIMIT m
\end{sqlcode}
If necessary, checks can be added that indeed $m$ rows have been selected.


\subsection{Generating a Random Variate According to a Discrete Probability Distribution}

In practice, probability distributions are often induced by weights (that are not necessarily normalized to add up to 1). The following algorithm is a special case of the ``unequal probability sampling plan'' proposed by \textcite{C82a}. Its idea is very similar to reservoir sampling \cite{MB83a}.

\subsubsection{Formal Description}

\begin{algorithm}[DiscreteSample$(A, w)$] \label{alg:DiscreteSample}
\alginput{Finite set $A$, Mapping $w$ of each element $a \in A$ to its weight $w[a] \geq 0$}
\algoutput{Random element $\Sample \in A$ sampled according to distribution induced by $w$}
\begin{algorithmic}[1]
	\State $W \set 0$
	\For{$a \in A$}
		\State $W \set W + w[a]$ \label{alg:DiscreteSample:UpdateWeight}
		\With{probability $\frac{w[a]}{W}$} \label{alg:DiscreteSample:Prob}
			\State $\Sample \set a$ \label{alg:DiscreteSample:SetSample}
		\EndWith
	\EndFor
\end{algorithmic}
\end{algorithm}

\begin{description}
	\item[Runtime] $O(n)$, single-pass streaming algorithm
	\item[Space] $O(1)$, constant memory
	\item[Correctness]
		Let $a_1, \dots, a_n$ be the order in which the algorithm processes the elements. Denote by $\Sample_t$ the value of $\Sample$ at the end of iteration $t \in \Nupto n$. We prove by induction over $t$ that it holds for all $i \in \Nupto t$ that $\Pr[\Sample_t = a_i] = \frac{w[a_i]}{W_t}$ where $W_t := \sum_{j=1}^t w[a_j]$.
			
		The base case $t = 1$ holds immediately by lines~\ref{alg:DiscreteSample:Prob}--\ref{alg:DiscreteSample:SetSample}. To see the induction step $t - 1 \to t$, note that $\Pr[\Sample_t = a_t] = \frac{w[a_t]}{W_t}$ (again by lines~\ref{alg:DiscreteSample:Prob}--\ref{alg:DiscreteSample:SetSample}) and that for all $i \in \Nupto{t-1}$
		\begin{align*}
			\Pr[\Sample_t = a_i]
			=	\Pr[\Sample_t \neq a_t] \cdot \Pr[\Sample_{t-1} = a_i]
			\stackrel{\text{IH}}{=}
				\left(1 - \frac{w[a_t]}{W_t}\right) \cdot \frac{w[a_i]}{W_{t-1}}
			=	\frac{w[a_i]}{W_t}
			\SiM.
		\end{align*}
	\item[Scalability]
		The algorithm can easily be transformed into a divide-and-conquer algorithm, as shown in the following.
\end{description}

\begin{algorithm}[RecursiveDiscreteSample$(A_1, A_2, w)$] \label{alg:RecursiveDiscreteSample}
\alginput{Disjoint finite sets $A_1, A_2$, Mapping $w$ of each element $a \in A_1 \cup A_2$ to its weight $w[a] \geq 0$}
\algoutput{Random element $\Sample \in A_1 \cup A_2$ sampled according to distribution induced by $w$}
\begin{algorithmic}[1]
	\State $\tilde A \set \emptyset$
	\For{$i = 1,2$}
		\State $\Sample_i \set \texttt{DiscreteSample}(A_i, w)$
		\State $\tilde A \set \tilde A \cup \{ \Sample_i \}$
		\State $\tilde w[\Sample_i] \set \sum_{a \in A_i} w[a]$
	\EndFor
	\State $\Sample \set \texttt{DiscreteSample}(\tilde A, \tilde w)$
\end{algorithmic}
\end{algorithm}

\paragraph{Correctness}

Define $W_i := \sum_{a \in A_i} w[a]$. Let $a \in A_i$ be arbitrary. Then $\Pr[\Sample = a] = \Pr[\Sample_i = a] \cdot \Pr[\Sample \in A_i] = \frac{w[a]}{W_i} \cdot \frac{W_i}{W} = \frac{w[a]}{W}.$

\paragraph{Numerical Considerations}

\begin{itemize}
	\item When Algorithm~\ref{alg:DiscreteSample} is used for large sets $A$, line~\ref{alg:DiscreteSample:UpdateWeight} will eventually add two numbers that are very different in size. Compensated summation should be used \cite{ORO05a}.
\end{itemize}

\subsubsection{Implementation as User-Defined Aggregate}

Algorithm~\ref{alg:DiscreteSample} is implemented as the user-defined aggregate \symlabel{discrete\_sample}{sym:discrete_sample}. Data-parallelism is implemented as in Algorithm~\ref{alg:RecursiveDiscreteSample}.

\paragraph{Input} The aggregate expects the following arguments:

\begin{center}
	\begin{tabularx}{\linewidth}{lXl}
		\toprule%
		\textbf{Column} & \textbf{Description} & \textbf{Type}
		\\\otoprule
		\texttt{id} &
		Row identifier, each row corresponds to an $a \in A$. There is no need to enforce uniqueness. If an identifier occurs multiple times, the probability of sampling this identifier is proportional to the sum of its weights. &
		integer
		\\\midrule
		\texttt{weight} &
		weight for row, corresponds to $w[a]$ &
		floating-point
		\\\bottomrule
	\end{tabularx}
\end{center}
%
While it would be desirable for \texttt{id} to be of generic type, this would also require a generic transition type (see below). However, this is currently not supported in PostgreSQL and Greenplum.

\paragraph{Output}

Value of column \texttt{id} in row that was selected.

\paragraph{Components}

\begin{itemize}
	\item Transition State:
		\begin{center}
			\begin{tabularx}{\linewidth}{lXl}
				\toprule%
				\textbf{Field Name} & \textbf{Description} & \textbf{Type}
				\\\otoprule
				\texttt{weight\_sum} &
				corresponds to $W$ in Algorithm~\ref{alg:DiscreteSample} &
				floating-point
				\\\midrule
				\texttt{sample\_id} &
				corresponds to $\Sample$ in Algorithm~\ref{alg:DiscreteSample}, takes value of column \texttt{id} &
				integer
				\\\bottomrule
			\end{tabularx}
		\end{center}
	\item Transition Function (\texttt{state, id, weight}): Lines~\ref{alg:DiscreteSample:UpdateWeight}--\ref{alg:DiscreteSample:SetSample} of Algorithm~\ref{alg:DiscreteSample}
	\item Merge Function (\texttt{state1, state2}): It is enough to call the transition function with arguments (\texttt{state1, state2.sample\_id, state2.weight\_sum})
\end{itemize}

\paragraph{Tool Set}

While the user-defined aggregate is simple enough to be implemented in plain SQL, we choose a C++ implementation for performance. Moreover, a C++ implementation allows us to reuse the existing compensated-sum component. \todo{Not documented yet.}


\chapter[k-Means Clustering]{$k$-Means Clustering}

% Abstract. What is the problem we want to solve?
Clustering refers to the problem of partitioning a set of objects into homogeneous subsets, i.e., such that objects in the same group have \emph{similar} properties. In k-means clustering,
one is given $n$ points $x_1, \dots, x_n \in \R^d$, and the goal is
to position $k$ centroids $c_1, \dots, c_k \in \R^d$ so that the sum of squared
distances between each point and its closest centroid is minimized. A cluster
is identified by its centroid and consists of all points for which this
centroid is closest. Formally, we wish to minimize the following objective
function:
\begin{gather*}
    (c_1, \dots, c_k) \mapsto \sum_{i=1}^n \min_{j=1}^k \dist(x_i, c_j)^2
\end{gather*}

\section{Overview of Algorithms} \label{sec:kmeans:Algorithms}

% Explain the algorithm at a high-level -- do not talk about specific variations or implementation details. Give some theoretical background: Is the problem hard? What results can we expect?
$k$-means clustering is NP-hard in general Euclidean space (even for just two clusters) \cite{ADH09a} and for a general number of clusters (even in the plane) \cite{MNV10a}. However, the local-search heuristic proposed by \citeauthor{L82a}~\cite{L82a} performs reasonably well in practice. In fact, it is so ubiquitous today that it is often referred to as the standard algorithm or even just the $k$-means algorithm. At a high level, it works as follows:

\begin{enumerate}
	\item Seeding phase: Find initial positions for $k$ centroids $c_1, \dots, c_k$.
	\item Assign each point $x_1, \dots, x_n$ to its closest centroid. \label{enum:kmeans_abstract_points}
	\item Reposition each centroid to the barycenter (mean) of all points assigned to it.
	\item If convergence has been reached, stop. Otherwise, goto \eqref{enum:kmeans_abstract_points}.
\end{enumerate}

Since the value of the objective function decreases in every step, and there is only a finite number of clusterings, the algorithm is guaranteed to converge to a local minimum \cite[Section~16.4]{CS08a}. While it is known that there are instances for which the $k$-means algorithm takes exponentially many steps~\cite{V09a}, it has been shown that $k$-means has polynomial smoothed complexity \cite{AMR09a}---thus giving some theoretical explanation for good results in practice. With a clever seeding technique, $k$-means is moreover $O(\log k)$-competitive \cite{AV07a}.


\subsection{Algorithm Variants}

% Give an overview and references to variations that exist for this algorithm.
\paragraph{Seeding}

The quality of $k$-means is highly influenced by the choice of the seeding \cite{AV07a}. The following is a non-exhaustive list of options:
\begin{enumerate}
	\item Manual: User-specified list of initial centroid positions.
	\item Uniformly at random: Choose the $k$ centroids uniformly at random among the point set
	\item \texttt{$k$-means++}: Perform seeding so that the objective function is minimized in expectation \cite{AV07a}
	\item Use a different clustering algorithm for determining the seeding \cite{MNU00a}
	\item Run $k$-means with multiple seedings and choose the clustering with lowest cost
\end{enumerate}

\paragraph{Repositioning}

Most $k$-means formulations in textbooks do not detail the case where a centroid has no points assigned to it. It is an easy observation that moving a stray centroid in this case can only decrease the objective function. This can be done in a simple way (move onto a random point) or more carefully (e.g., move so that the objective function is minimized in expectation).

\paragraph{Convergence Criterion}

There are several reasonable convergence criteria. E.g., stop when:
\begin{enumerate}
	\item The number of repositioned centroids is below a given threshold
	\item The change in the objective drops below a given threshold
	\item The maximum number of iterations has been reached
	\item See, e.g., \textcite[Section~16.4]{CS08a} for more options.
\end{enumerate}

\paragraph{Variable Number of Clusters}

The number of clusters $k$ could be determined by the seeding algorithm (instead of being a parameter) \cite{MNU00a}. Strictly speaking, however, the algorithm should not be called $k$-means in this case.


\section{Seeding Algorithms}

In the following, we describe the seeding methods to be implemented for MADlib.

\subsection{Uniform-at-random Sampling}

Uniform-at-random sampling just uses the algorithms described in Section~\ref{sec:SampingWOReplacement}.

\subsection[k-means++]{$k$-means++}

\texttt{$k$-means++} seeding \cite{AV07a} starts with a single centroid chosen randomly among the input points. It then iteratively chooses new centroids from the input points until there is a total of $k$ centroids. The probability for picking a particular point is proportional to its minimum squared distance to any existing centroid. Intuitively, \texttt{$k$-means++} favors seedings where centroids are spread out over the whole range of the input points, while at the same time not being too susceptible to outliers. 

\subsubsection{Formal Description}

\begin{algorithm}[$k$-means++$(k, P, \dist)$] \label{alg:kmeans++}
\alginput{Number of desired centroids $k$, set $P$ of points in $\R^d$, metric $\dist$}
\algoutput{Set of centroids $C$}
\begin{algorithmic}[1]
	\State $C \set \{ \text{initial centroid chosen uniformly at random from } P \}$ \label{alg:kmeanspp:firstCentroid}
	\For{$i \to 1, \dots, k - 1$} \label{alg:kmeans++:for}
		\State $C \set C \cup \{ \text{random $p \in P$ with probability proportional to }\min_{c \in C} \dist(p,c)^2 \}$ \label{alg:kmeanspp:nextcentroid}
	\EndFor
\end{algorithmic}
\end{algorithm}

\begin{description}
	\item[Runtime] A naive implementation needs $\Theta(k^2 n)$ distance calculations, where $n = |P|$. A single distance calculation takes $O(d)$ time.
	\item[Space] Store $k$ centroids.
	\item[Subproblems]
		Existing subroutines should be used for:
		\begin{itemize}
			\item Line~\ref{alg:kmeanspp:firstCentroid}: Sample uniformly at random
			\item Line~\ref{alg:kmeanspp:nextcentroid}: Sample according to a discrete probability distribution.
		\end{itemize}
		In each case, see Section~\ref{sec:SampingWOReplacement}.
\end{description}

The runtime can be reduced by a factor of $k$ if we store, for each point $p \in P$, the distance to its closest centroid. Then, each iteration only needs $n$ distance calculations (i.e., only between the most recently added centroid and all points). In total, these are $\Theta(k n)$ distance calculations. Making this idea explicit leads to the following algorithm.

\begin{algorithm}[$k$-means++-ext$(k, P, \dist)$] \label{alg:kmeans++ext}
\alginput{Number of desired centroids $k$, set $P$ of points in $\R^d$, metric $\dist$}
\algoutput{Set of centroids $C$}
\algprecond{For all $p \in P: \delta[p] = \infty$}
\begin{algorithmic}[1]
	\State $\mathit{lastCentroid} \set \ref{sym:discrete_sample}(P, 1)$ \label{alg:kmeans++ext:firstCentroid} \Comment{1 denotes the constant mapping $p \mapsto 1$}
	\State $C \set \{ \mathit{lastCentroid} \}$
	\For{$i \to 1, \dots, k - 1$} \label{alg:kmeans++ext:for}
		\For{$p \in P$} \label{alg:kmeans++ext:pointLoop}
			\If{$\dist(p, \mathit{lastCentroid}) < \delta[p]$}
				\State $\delta[p] \set \dist(p, \mathit{lastCentroid})$
			\EndIf
		\EndFor
		\State $\mathit{lastCentroid} \set \ref{sym:discrete_sample}(P, \delta^2)$ \Comment{$\delta^2$ denotes the mapping $p \mapsto \delta[p]^2$} \label{alg:kmeans++ext:nextCentroid}
		\State $C \set C \cup \{ \mathit{lastCentroid} \}$
	\EndFor
\end{algorithmic}
\end{algorithm}

\begin{description}
	\item[Tuning] \label{kmeans++ext:tuning} The inner for-loop in line~\ref{alg:kmeans++ext:pointLoop} could be combined with \ref{sym:discrete_sample} in line~\ref{alg:kmeans++ext:nextCentroid}. With this improvement, only one pass over $P$ is necessary.
	\item[Runtime] $O(dkn)$ as explained before.
	\item[Space] Store $k$ centroids and $n$ distances.
	\item[Scalability] The outer for-loop is inherently sequential because the random variates in each iteration depend on all previous iterations. The inner loop, however, can be executed with data parallelism.
\end{description}

\subsubsection{Implementation as User-Defined Function}

Algorithm~\ref{alg:kmeans++ext} is implemented as the user-defined function \symlabel{kmeans\_pp}{sym:kmeans++}.

\paragraph{In- and Output} The function expects the following arguments:

\begin{center}
	\begin{tabular}{lll}
		\toprule%
		\textbf{Name} & \textbf{Description} & \textbf{Type}
		\\\otoprule
		\texttt{k} &
		Number of centroids &
		integer
		\\\midrule
		\texttt{dist} &
		Metric to use &
		functor
		\\\midrule
		\texttt{points} &
		Relation containing the points as rows &
		relation
		\\\bottomrule
	\end{tabular}
\end{center}
%
Here, \texttt{points} refers to a relation with at least the following columns.
%
\begin{center}
	\begin{tabularx}{\linewidth}{rlXl}
		\toprule%
		& \textbf{Column} & \textbf{Description} & \textbf{Type}
		\\\otoprule
		In &
		\texttt{id} &
		Row identifier, each row corresponds to a point $p \in P$ &
		integer
		\\\midrule
		In &
		\texttt{coords} &
		Point coordinates, i.e., the point $p$ &
		sparse vector
		\\\midrule
		Temp &
		\texttt{centroid\_dist} &
		distance to closest centroid, corresponds to $\delta[p]$ in Algorithm~\ref{alg:kmeans++ext} &
		floating-point
		\\\bottomrule
	\end{tabularx}
\end{center}
%
The return value of \ref{sym:kmeans++} is an array of centroids of type sparse vector.

\paragraph{Components} The set of centroids $C$ is stored as an array of sparse vectors. Algoritm~\ref{alg:kmeans++ext} can be translated into SQL in a straightforward fashion. Unfortunately, PostgreSQL and Greenplum do not support an \texttt{UPDATE} statement that returns an aggregate (i.e., \ref{sym:discrete_sample}), so we cannot implement the aforementioned tuning.

The inner for-loop in line~\ref{alg:kmeans++ext:pointLoop} becomes:
\begin{sqlcode}
	UPDATE points
	SET centroid_id = last_centroid,
	    centroid_dist = dist(coords, last_centroid)
	WHERE
	    dist(coords, last_centroid) < centroid_dist
\end{sqlcode}
Lines~\ref{alg:kmeans++ext:firstCentroid} and \ref{alg:kmeans++ext:nextCentroid} are implemented using \ref{sym:discrete_sample}. That is, \texttt{last\_centroid} is set to the output of:
\begin{sqlcode}
	SELECT discrete_sample(id, centroid_dist^2) FROM points
\end{sqlcode}

\subsubsection{Historical Implementations}

Implementation details and big-data heuristics that were used in previous versions of MADlib are documented here for completeness.

\begin{description}
	\item[v0.2.1beta and earlier] In lines~\ref{alg:kmeanspp:firstCentroid} and \ref{alg:kmeanspp:nextcentroid} of Algorithm~\ref{alg:kmeans++} use a random sample $P' \subsetneq P$.
	
		Here $P'$ will be a new random sample in each iteration. Under the a-priori assumption that a random point belongs to any of the $k$ (unknown) clusters with equal probability, sample enough points so that with high probability (e.g., $p = 0.999$) there is a point from each of the $k$ clusters.
	
		This is the classical occupancy problem (also called balls-into-bins model) \cite{F68a}: Throwing $r$ balls into $k$ bins, what is the probability that no bin is empty? The exact value is
		\begin{align*}
			u(r, k) = k^{-r} \sum_{i=0}^k (-1)^i \binom ki (k - i)^r
			\SiM.
		\end{align*}
		
		For $r,k \to \infty$ so that $r/k = O(1)$ we have the limiting form $u(r,k) \to (1 - e^{-r/k})^k =: \widetilde u(r, k)$. Rearranging $\widetilde u(r, k) > p$ gives $-\log(1 - \sqrt[k]p) \cdot k < r$. The smallest $r$ satisfying this inequality is chosen as the size of the sample set.
\end{description}

\section[Standard algorithm for k-means clustering]{Standard algorithm for $k$-means clustering}

The standard algorithm has been outlined in Section~\ref{sec:kmeans:Algorithms}. The formal description and our implementation are given below.

\subsection{Formal Description}

\begin{algorithm}[$k$-means$(k, P, \dist)$] \label{alg:kmeans}
\alginput{Number of desired centroids $k$, seeding strategy $\mathit{Seeding}$, set $P$ of points, metric $\dist$,\\convergence strategy $\mathit{Convergence}$}
\algoutput{Set $C$ of final means, mapping $m$ of each $p \in P$ to closest mean $m[p]$}
\algprecond{$m = $ empty mapping, $i = 0$}
\begin{algorithmic}[1]
	\State $C \set \mathit{Seeding}(k, P, \dist)$ \label{alg:kmeans:Seed}
	\Repeat
		\State $i \set i + 1$
		\State $m_\text{old} \set m$
		\For{$p \in P$} \label{alg:kmeans:ForReassign}
			\State $m[p] \set \arg\min_{c \in C} \dist(p, c)$ \Comment{Break ties arbitrarily}
		\EndFor
		\State $C \set \bigcup_{c \in C} \{ \operatorname{avg}_{p \in P \mid m[p] = c} p \}$ \label{alg:kmeans:MoveCentroids}
		\State $C \set C \cup \mathit{Seeding}(k - |C|, P, \dist)$ \label{alg:kmeans:Reseed}
	\Until{$Convergence(m_\text{old}, m, i)$} \label{alg:kmeans:ConvergenceCond}
\end{algorithmic}
\end{algorithm}

\begin{description}
	\item[Runtime] See discussion in Section~\ref{sec:kmeans:Algorithms}.
	\item[Space] Store the $k$ centroids plus the closest centroid for each of the $n$ points
	\item[Scalability] The outer loop is inherently sequential. The inner loop is data-parallel (provided that the set $C$ is available on all computation nodes).
	\item[Heuristics] The following are straightforward heuristic modifications that improve performance but that we expect to have little impact on the quality of output:
		\begin{itemize}
			\item Random Sampling: In lines~\ref{alg:kmeans:Seed} and \ref{alg:kmeans:Reseed} use a random sample $P' \subsetneq P$.
		\end{itemize}
\end{description}

\subsection{Implementation as User-Defined Function}

Algorithm~\ref{alg:kmeans} is implemented as the user-defined function \symlabel{kmeans}{sym:kmeans}. Since specifying higher-order arguments is problematic in SQL, we choose to not make the convergence criterion a function argument but instead settle for parameters for the most typical criteria.

\paragraph{In- and Output} The function expects the following arguments:

\begin{center}
	\begin{tabularx}{\linewidth}{lXl}
		\toprule%
		\textbf{Name} & \textbf{Description} & \textbf{Type}
		\\\otoprule
		\texttt{k} &
		Number of centroids &
		integer
		\\\midrule
		\texttt{seeding} &
		Seeding strategy &
		functor
		\\\midrule
		\texttt{dist} &
		Metric &
		functor
		\\\midrule
		\texttt{points} &
		Relation containing the points as rows &
		relation
		\\\midrule
		\texttt{max\_iterations} &
		Convergence criterion: Maximum number of iterations &
		integer
		\\\midrule
		\texttt{conv\_level} &
		Convergence criterion: Convergence is reached if the fraction of points being reassigned to another centroid drops below \texttt{conv\_level} &
		floating-point
		\\\midrule
		\texttt{seeding\_sample} &
		Relative size of sample used for seeding, i.e., the size of $P'$ (relative to $P$) passed to $\mathit{Seeding}$ in lines~\ref{alg:kmeans:Seed} and \ref{alg:kmeans:Reseed}  &
		floating-point
		\\\bottomrule
	\end{tabularx}
\end{center}
%
Here, \texttt{points} refers to a relation with at least the following columns.
%
\begin{center}
	\begin{tabularx}{\linewidth}{rlXl}
		\toprule%
		& \textbf{Column} & \textbf{Description} & \textbf{Type}
		\\\otoprule
		In &
		\texttt{id} &
		Row identifier, each row corresponds to a point $p \in P$ &
		integer
		\\\midrule
		In &
		\texttt{coords} &
		Point coordinates, i.e., the point $p$ &
		sparse vector
		\\\midrule
		Out &
		\texttt{centroid\_id} &
		id of closest centroid, corresponds to $m[p]$ in Algorithm~\ref{alg:kmeans} &
		integer
		\\\bottomrule
	\end{tabularx}
\end{center}
%
The return value of \ref{sym:kmeans++} is an array of centroids of type sparse vector.

\paragraph{Components}

The set of centroids $C$ is stored as an array of sparse vectors \texttt{centroids}. Algoritm~\ref{alg:kmeans} can be translated into SQL in a straightforward fashion. The point-reassignment phase in the inner for-loop in line~\ref{alg:kmeans:ForReassign} becomes
\begin{sqlcode}
	UPDATE points SET centroid_id = closest_centroid(centroids, coords, dist)
\end{sqlcode}
Here, \texttt{closest\_centroid} is a simple function that takes an array of sparse vectors, a sparse vector, and a metric. It returns the index of the closest vector in the array. \todo{With higher-order constructs we can avoid the need for specialized functions and express this much more elegantly. While SQL makes this hard, we should still think about ways that make this possible.}

Likewise, moving the centroids in line~\ref{alg:kmeans:MoveCentroids} is implemented as follows.
\begin{sqlcode}
	CREATE TEMPORARY TABLE updated_centroids AS
	SELECT centroid_id, avg(coords) FROM points GROUP BY centroid_id
\end{sqlcode}
Since there might be stray centroids now, \texttt{updated\_centroids} may have less than $k$ rows. As in line~\ref{alg:kmeans:Reseed}, we therefore call the $\texttt{seeding}$ function again and store the returned array of centroids in \texttt{reseeded}. Unfortunately, some fiddling is necessary to assign the vacant \texttt{centroid\_id}s among the the centroids in \texttt{reseeded}:
\begin{sqlcode}
	INSERT INTO updated_centroids
	SELECT centroid_id, reseeded[sub_id] AS coords
	FROM (
	    SELECT *, row_number() OVER () AS sub_id
	    FROM (
	        SELECT generate_series(1,k) AS centroid_id
	        EXCEPT
	        SELECT centroid_id FROM updated_centroids
	    ) vacant_ids
	) vacant_ids_with_sub_ids
\end{sqlcode}
The centroids are then converted to an array again:
\begin{sqlcode}
	SELECT array_agg(coords ORDER BY centroid_id) FROM updated_centroids
\end{sqlcode}


\section{Variants}

\subsection{Canopy clustering}

\todo{Unfinished. To be completed.}
\textcite{MNU00a}

\begin{algorithm}[Canopy$(t_1, t_2, P, \dist)$]
\alginput{Parameters $t_1 > t_2$, set $P$ of points, metric $\dist$}
\algoutput{Set of canopies (points) $C$,\\Mapping $D$ of each point $p \in P$ to the set $D[p] \subseteq C$ of canopies that contain $p$}
\begin{algorithmic}[1]
	\State $C \set \emptyset$
	\For{$p \in P$}
		\If{$\min_{c \in C} \dist(p, c) \geq T_2$}
			\State $C \set C \cup \{ p \}$
		\EndIf
	\EndFor
	\For{$p \in P$}
		\State $D[p] \set \{ c \in C \mid \dist(p, c) < T_1 \}$
	\EndFor
\end{algorithmic}
\end{algorithm}

\paragraph{Properties:}
\begin{itemize}
	\item Runtime: Naive implementation needs $\Theta(k^2 n)$ distance calculations, where $k = |C|, n = |P|$. (But $k$ is not a parameter!)
	\item Space: $O(dn)$. Without further knowledge, no better bound can be given.
\end{itemize}

\printbibliography[maxnames=20]

\end{document}
